\section{Méthode de résolution}
Afin de trouver la perte de pression dans la conduite, il convient de définir les équations qui régissent le problème.
\subsection{Équations du problème}
Les différentes équations dont on dispose sont :

des équations d'état lors que le mélange est à saturation :
\begin{align}
    T_{sat} &= T_{sat}(P) \\
    h_{l,sat} &= h_{l,sat}(T) = h_{l,sat}(P) \\
    h_{v,sat} &= h_{v,sat}(T) = h_{v,sat}(P)
\end{align}

Des lois de comportement, notamment pour calculer la perte de pression
\begin{equation}
     -\pder[p]{z} = \left(\pder[p]{z} \right)_{a} + \left(\pder[p]{z} \right)_{f} + \left(\pder[p]{z} \right)_{g}
\end{equation}

Que l'on peut écrire comme :
\begin{equation}
    -\pder[p]{z} = \underbrace{\left[\overbrace{\cancel{\frac{\partial}{\partial \tau} G}}^{\text{Ecoul. permanent}}+\frac{\partial}{\partial z} G^{2} v^{\prime}\right]}_{\text{Acceleration}}+\underbrace{\frac{4 \tau_{w}}{D}}_{\text{Frottements}}+\underbrace{\rho_{m} g}_{\text{Gravitation}}
%    -\pder[p]{z} = \phi_{l0}^2 \times \left( \pder[p]{z} \right)_{1\varphi}
\end{equation}

Avec :
\begin{equation}
    v^{\prime}=\frac{x^{2}}{\varepsilon \rho_{v}}+\frac{(1-x)^{2}}{(1-\varepsilon) \rho_{l}}
\end{equation}

Le terme de frottement pouvant être écrit comme la perte de pression (par frottement) si l'écoulement était seulement liquide et une correction le \og multiplicateur diphasique \fg{}. Ce dernier sera obtenu par la corrélation de \textsc{Friedel}.
\begin{equation}
    \left(\pder[p]{z} \right)_{f} = f\frac{G^2}{2 \rho_l}\frac{1}{D} \phi_{lo}^2
\end{equation}

Le facteur de frottement $f$ est défini par :
\begin{equation}
    f = 0.316 \left[\text{Re}_L\right]^{-0.25} = 0.316 \left[\frac{(1-x)G D}{\mu_l}\right]^{-0.25} 
\end{equation}

Une relation entre le taux de vide moyen $\varepsilon$ et le titre de l'écoulement $x$ provenant du modèle à vitesses séparées. Il est accompagné de lois de comportements pour fixer les constantes de corrélation :
\begin{equation}
        \frac{1}{\varepsilon} = \frac{\rho_g}{\hat{x} G} <V_{gj}>_{2g} + C_0\left(\frac{\rho_g}{\rho_l}\frac{1-\hat{x}}{\hat{x}} + 1\right) 
\end{equation}
\begin{align*}
         C_0 &\textnormal{ fonction des paramètres de l'écoulement}\\
        <V_{gj}>_{2g} &\textnormal{ fonction des paramètres de l'écoulement}   
\end{align*}

Des lois de conservation, notamment de la masse :
\begin{equation}
    \pder[G]{z} = 0
\end{equation}

et de l'énergie du mélange :
\begin{equation}
    \pder{z} \left[ h_{l,sat}(T_{(z)})+x_{(z)}h_{vl,sat}(T_{(z)})\right] = \frac{4}{DG}q''
\end{equation}

où $q''$ est le flux de chaleur ajouté en paroi obtenu en supposant la puissance thermique $P_{th}$ apportée uniformément répartie sur la longueur chauffée $L_x$ du cylindre : $q'' = \frac{P_{th}}{\pi DL_c}$.\\ \par
On peut donc écrire la perte de pression que nous allons évaluer comme :
\begin{equation}
     -\pder[p]{z} = G^2\pder{z}\left[\frac{x^{2}}{\varepsilon \rho_{v}}+\frac{(1-x)^{2}}{(1-\varepsilon) \rho_{l}} \right] + f\frac{G^2}{2 \rho_l}\frac{1}{D} \phi_{lo}^2 + \rho_m g
\end{equation}
\subsection{Zone de saturation\label{subs:sat}}
Comme nous négligeons la zone où l'ébullition est sous-refroidie (\textbf{H1}), il nous faut calculer l'endroit où le fluide est à température de saturation. On peut écrire que la puissance nécessaire pour y arriver est :
\begin{equation}
    \dot{Q}_{sat} = \dot{m} Cp \Delta T
\end{equation}
Le terme $Cp$ pouvant être pris pour de l'eau à saturation, et le terme $\Delta T$ correspond à la différence entre la température de saturation du fluide à la pression donnée et la température d'entrée.\\ 
Comme on considère que la conduite est chauffée uniformément, le rapport entre puissance nécessaire pour arriver à saturation et puissance totale thermique ajoutée $P_{th}$ est égale au rapport entre $L_{sat}$ et longueur totale chauffée $L_c$. On a alors :
\begin{equation}
    L_{sat}= \frac{ \dot{Q}_{sat}\times L_c}{P_{th}}
\end{equation}
\par
\subsection{Fonctionnement du programme}
Dans notre étude, nous connaissons seulement la pression de sortie et la température d'entrée. La section \ref{subs:sat} nous a permis de trouver la position à laquelle le fluide est à saturation. C'est à partir de cette zone que la perte de pression va avoir lieu. \\ \par
Dans un premier temps on initialise la perte de pression comme nulle (ie. la pression d'entrée est égale à la pression de sortie) et on réalise les calculs avec les différentes corrélations. \\
Comme on connaît la pression de sortie, on ajoute le $\Delta P$ trouvé à la première itération à la pression d'entrée et on itère jusqu'à avoir une différence entre les itérations $i$ et $i+1$ suffisamment petite.