\section{Méthode de résolution\label{section:res}}
Afin de trouver la perte de pression dans la conduite, il convient de définir les équations qui régissent le problème.
\subsection{Équations du problème}
Les différentes équations dont on dispose sont :

des équations d'état lors que le mélange est à saturation :
\begin{align}
    T_{sat} &= T_{sat}(P) \\
    h_{l,sat} &= h_{l,sat}(T) = h_{l,sat}(P) \\
    h_{v,sat} &= h_{v,sat}(T) = h_{v,sat}(P)
\end{align}

Des lois de comportement, notamment pour calculer la perte de pression
\begin{equation}
     -\pder[p]{z} = \left(\pder[p]{z} \right)_{a} + \left(\pder[p]{z} \right)_{f} + \left(\pder[p]{z} \right)_{g}
\end{equation}

Que l'on peut écrire comme :
\begin{equation}
    -\pder[p]{z} = \underbrace{\left[\overbrace{\cancel{\frac{\partial}{\partial \tau} G}}^{\text{Ecoul. permanent}}+\frac{\partial}{\partial z} G^{2} v^{\prime}\right]}_{\text{Acceleration}}+\underbrace{\frac{4 \tau_{w}}{D}}_{\text{Frottements}}+\underbrace{\rho_{m} g}_{\text{Gravitation}}
%    -\pder[p]{z} = \phi_{l0}^2 \times \left( \pder[p]{z} \right)_{1\varphi}
\end{equation}

Avec :
\begin{equation}
    v^{\prime}=\frac{x^{2}}{\varepsilon \rho_{v}}+\frac{(1-x)^{2}}{(1-\varepsilon) \rho_{l}}
\end{equation}

Etant donné la complexité de gérer les cas $\varepsilon \rightarrow 0$ et $\varepsilon \rightarrow 1$ dans l'équation précédente, on a utilisé une équation dérivée du modèle homogène pour $v'$ :

\begin{equation}
    v^{\prime} = \frac{1}{\rho_m} = \frac{1}{\varepsilon \rho_v + (1-\varepsilon)\rho_l }
\end{equation}

Le terme de frottement pouvant être écrit comme la perte de pression (par frottement) si l'écoulement était seulement liquide et une correction le \og multiplicateur diphasique \fg{}. Ce dernier sera obtenu par la corrélation de \textsc{Friedel} rappelée dans \cite{revellinAdiabaticTwophaseFrictional2007}.
\begin{equation}
    \left(\pder[p]{z} \right)_{f} = f\frac{G^2}{2 \rho_l}\frac{1}{D} \phi_{lo}^2
\end{equation}

Comme rappelé par \cite{revellinAdiabaticTwophaseFrictional2007}, le facteur de frottement $f$ est défini par :
\begin{equation}
    f = 0.316 \left[\text{Re}_L\right]^{-0.25} = 0.316 \left[\frac{G D}{\mu_l}\right]^{-0.25} 
\end{equation}
On peut donc écrire la perte de pression que nous allons évaluer comme :
\begin{equation}
     -\pder[p]{z} = G^2\pder{z}\left[\frac{1}{\varepsilon \rho_v + (1-\varepsilon)\rho_l } \right] + f\frac{G^2}{2 \rho_l}\frac{1}{D} \phi_{lo}^2 + \left[ \varepsilon \rho_v + (1-\varepsilon)\rho_l  \right] g
\end{equation}


On dispose ensuite d'une relation entre le taux de vide moyen $\varepsilon$ et le titre de l'écoulement $x$ provenant du modèle à vitesses séparées. Il est accompagné de lois de comportements pour fixer les constantes de corrélation :
\begin{equation}
        \frac{1}{\varepsilon} = \frac{\rho_g}{\hat{x} G} <V_{gj}>_{2g} + C_0\left(\frac{\rho_g}{\rho_l}\frac{1-\hat{x}}{\hat{x}} + 1\right) 
\label{eq:DriftFluxModel}
\end{equation}
\begin{align*}
         C_0 &\textnormal{ fonction des paramètres de l'écoulement}\\
        <V_{gj}>_{2g} &\textnormal{ fonction des paramètres de l'écoulement}   
\end{align*}

En pratique on l'utilisera de deux manières. Dans l'algorithme classique, on s'en sert pour obtenir $\varepsilon$ à partir de $x$ et $p$ en inversant l'équation \ref{eq:DriftFluxModel}. Dans le PINN, on utilise plutôt la forme suivante où on retourne $x$ à partir d'un set initial de $\varepsilon,x$ et $p$ :

\begin{equation}
    x = \varepsilon \left[ \frac{\rho_g}{G}<V_{gj}>_{2g} + C_0 \left( \frac{1-x}{\rho_l}\rho_g + x\right) \right]
\end{equation}

Il faut bien noter que $C_0$ et $<V_{gj}>$ dépendent des solutions $\varepsilon,x$ et $p$ ainsi que des propriétés thermodynamiques qui varient elles-même avec les solutions, d'où la complexité du problème.\\ 

Enfin on utilise des lois de conservation, notamment de la masse qui permet de fixer $G$ pour tout le problème :
\begin{equation}
    \pder[G]{z} = 0
\end{equation}

et de l'énergie du mélange :
\begin{equation}
    \pder{z} \left[ h_{l,sat}(T_{(z)})+x_{(z)}h_{vl,sat}(T_{(z)})\right] = \frac{4}{DG}q''
\label{eq:titre}
\end{equation}

où $q''$ est le flux de chaleur ajouté en paroi obtenu en supposant la puissance thermique $P_{th}$ apportée uniformément répartie sur la longueur chauffée $L_x$ du cylindre : $q'' = \frac{P_{th}}{\pi DL_c}$. On suppose que l'on travaille à saturation, où la pression $p$ et la température $T$ sont directement liées. Ainsi on n'a pas besoin de résoudre explicitement $T = T_{sat}(p_{(z)})$\\ \par


\subsection{Fonctionnement du programme}
\textcolor{red}{Dans notre étude, nous connaissons seulement la pression de sortie et la température d'entrée. La section \ref{subs:sat} nous a permis de trouver la position à laquelle le fluide est à saturation. C'est à partir de cette zone que la perte de pression va avoir lieu.} \\ \par
\textcolor{red}{Dans un premier temps on initialise la perte de pression comme nulle (ie. la pression d'entrée est égale à la pression de sortie) et on réalise les calculs avec les différentes corrélations.} \\
\textcolor{red}{Comme on connaît la pression de sortie, on ajoute le $\Delta P$ trouvé à la première itération à la pression d'entrée et on itère jusqu'à avoir une différence entre les itérations $i$ et $i+1$ suffisamment petite.}