\section{Présentation de la méthode PINN}

\subsection{Introduction et état de l'art}

Partir de la régression symbolique appliquée à la représentation/résolution de PDE.
"Dsicovering governing equations from data ...", S Brunton \cite{bruntonDiscoveringGoverningEquations2016}
Qui a amené à :
"Application of Sparse Identification ..." Corbetta \cite{corbettaApplicationSparseIdentification}
Puis Raissi \cite{raissiPhysicsinformedNeuralNetworks2019,raissiHiddenFluidMechanics2018}, Karniadakis, et récents papiers
Chap 5 "Solving Nonlinear and H-D PDE via DL", Al-Aradi \cite{al-aradiSolvingNonlinearHighDimensional}
PhD Rudy \cite{rudyComputationalMethodsSystem2019}


Ouvertures à d'autres sujets :
IFS \cite{raissiDeepLearningVortexinduced2019a}
High-Speed flows : \cite{maoPhysicsinformedNeuralNetworks2020}
Solid mechanics \cite{haghighatDeepLearningFramework2020,luExtractionMechanicalProperties2020}
Nano optics and metamaterals : \cite{chenPhysicsinformedNeuralNetworks2020}


\subsection{Principe de fonctionnement}

\subsection{Adaptation au problème du projet et limites}

