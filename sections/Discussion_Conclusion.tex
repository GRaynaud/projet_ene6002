\section{Discussion et conclusion}
D'après les différentes sections précédentes présentant les modèles et les méthodes de résolutions, ainsi que des graphes des résultats nous pouvons nous interroger sur les résultats obtenus.\\ \par

Nous avons réalisés pour les deux jeux de donnés des expériences, une résolution avec l'algorithme classique et l'algorithme PINN. Pour chacune de ces méthodes, les modèles de Chexal et Inoue on été testés. Nous avons ainsi 8 cas que nous pouvons comparer pour en tirer des conclusions.
\subsection{Commentaires sur les résultats obtenus}
Les premières comparaisons que nous pouvons faire sont entre les deux algorithmes. La principale différence que l'on remarque entre les résultats, est la discontinuité pour le taux de vide que l'on obtient en utilisant l'algorithme classique. La manière dont nous avons travaillé est de le contraindre égal à zéro dans toute la zone précédent la saturation (lorsque $x<0$), d'où le fait qu'il n'augmente qu'après. Cela a pour conséquence de créer une fonction dont la dérivée est non continue pour ce paramètre mais qui respecte mieux les critères du modèle.\\

De son coté, l'algorithme PINN a plus de mal à converger vers les bons résultats, la tendance semble correcte cependant (croissante et du même ordre de grandeur) ce qui nous rassure par rapport à l'implantation de notre algorithme. Il faut noter qu'en théorie la zone avant saturation ne pénalise pas les "bonnes" équations.\\ \par

Dans tous les cas que nous avons essayés, on peut voir que la pression correspond bien au jeux de donnés disponibles. 
Les tables \ref{PressureDrop} et \ref{ErrorPressureDrop} nous montre que la chute de pression est du même ordre de grandeur que ce qui était attendu quelque soit la méthode utilisée. Les écarts relatifs sont très faibles, et la solution trouvée à la même enveloppe que les données. Le changement de pente à l'abcisse qui correspond à $x=0$ correspond à l'hypothèse \textbf{H2}, le fait de négliger la perte de pression par accélération dans la zone sous-refroidie.\\

Comme précédemment, la méthode PINN cherche une fonction qui fait tendre les résidus vers 0, il est donc normal de ne pas obtenir exactement les mêmes résultats que pour la méthode classique. En revanche, on remarque que les solutions ne sont pas éloignées, et que la courbe possède une pente presque constante basée sur les résultats dans la zone de saturation.\\ \par
Enfin, pour ce qui est du titre de l'écoulement, les résultats semblent cohérents quand nous travaillons avec les mesures d'une même expérience. \\

L'algorithme PINN a plusieurs résultats difficiles à accepter, comme l'expérience 19 avec la corrélation d'Inoue (fig. \ref{fig:Result_PINN_19}). On a un titre qui n'est jamais nul, ce qui ne peut pas se produire étant donné que la première zone de la conduite est censée être sous-refroidie. Cela s'explique par le fait qu'on ne fixe nulle part  une condition à la limite pour $x$. Sa dérivée intervient dans plusieurs équations mais sa valeur même n'intervient implicitement (dans des propriétés thermodynamiques du mélange par exemple). Les résultats présentés fig. \ref{fig:Result_PINN_65BV} montrent que dans certains cas, la longueur à saturation a été déterminée sans connaissance de conditions aux limites autres que la pression de sortie.\\ \par

Pour tous les résultats obtenus, il semble cohérent de dire que ceux qui proviennent de l'algorithme classique sont bons. Ils correspondent aux données expérimentales et le titre de l'écoulement (dont nous n'avons pas de données de comparaison) est le même pour les deux corrélations utilisées.\\

L'algorithme PINN nécessite un choix de paramètres précis pour qu'il fonctionne et qu'il converge vers les bonnes valeurs. Une fois fonctionnel il est plus flexible que l'algorithme classique car il permet d'avoir toutes les propriétés de l'écoulement à toutes les positions de la conduite. Mais son implémentation plus complexe nécessite d'avoir plus de jeux de données ou bien comme nous l'avons fait un algorithme plus classique, en discrétisation pour pouvoir faire des comparaisons.

\subsection{Conclusion}

La résolution de ce problème par un algorithme classique a révélé des résultats convaincants. La partie la plus complexe étant l'implantation des modèles de corrélation, la partie principale de l'algorithme a montré sa robustesse et sa rapidité d'exécution. Les écarts de chute de pression avec les données expérimentales sont contenues sous les 10\% pour l'ensemble des cas étudiés et sont en accord avec les relevés ponctuels dont on dispose. En revanche il est difficile de dire si l'une des deux corrélations du modèle à écart de vitesse est meilleure que l'autre car les résultats différent selon les configurations expérimentales.\\

Implanter la méthode PINN fut un défi intéressant : avec de nombreux aspects pertinents d'un point de vue pratique (différentiation plus poussée et simple, modèles thermodynamiques, représentation continue..) il y a sûrement des pistes à approfondir. Néanmoins du fait du manque de régularité des modèles et du manque de recul sur ces méthodes, nous n'avons pas pu résoudre un certain nombre de points pour atteindre la précision et la robustesse escomptées.\\

Une perspective qu'il pourrait être intéressant de suivre pour canaliser la multitude de modèles de corrélations et leur complexité d'implémentation peut être cherchée du côté de modèles réduits conçus sous forme de réseaux de neurones "simples", bien plus mature dans ce domaine là, comme cela a été fait chez \cite{alvarezdelcastilloNewVoidFraction2012}. La perte en interprétabilité pourrait alors être compensée par un regain en efficacité et éventuellement un meilleur couplage avec un PINN.\\

\medskip

Les codes Python et données utilisées seront disponibles sous licence \href{https://choosealicense.com/licenses/gpl-3.0/}{GPLv3.0} dans le répertoire Github suivant :
\begin{center}
    \href{https://github.com/GRaynaud/projet_ene6002}{github.com/GRaynaud/projet\_ene6002}
\end{center}